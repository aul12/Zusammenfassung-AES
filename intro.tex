\chapter{Einführung}
Ein eingettetes System ist in einen technischen Kontext oder Prozess eingebettet.

Im wesentlichen kann ein eingebettetes System als ein Computer, der einen technischen Prozess steuert oder regelt, betrachtet werden.
\todo{Grafik}

\section{Architektur eines Eingebetteten Systems}
\subsection{Eigenschaften eines Eingebetteten Systems}
\begin{itemize}
    \item Enge Verzahnung zwischen Hard- und Software
    \item Strenge funktionale und zeitliche Randbedinungen
    \item Zusätzlich zum Prozessor wird I/O Hard- und Software benötigt
    \item Oftmals wird Anwendungsspezifische Hardware benötigt
\end{itemize}
$\Rightarrow$ Keine \glqq{}General-Purpose\grqq{} Lösung möglich

Zusätzliche Probleme:
\begin{itemize}
    \item Wenig Platz
    \item Nur beschränkte Energiekapazität
    \item System darf nicht warm werden
    \item Kostengünstig
\end{itemize}

\subsection{Zusätzliche Herausforderungen beim Entwurf}
Die Entwicklung eines eingebetteten Systems ist kein reines Software-Problem, zusätzlich muss beachtet werden:
\begin{itemize}
    \item Auswahl eines Prozessors, Signalprozessors, Microcontrollers
    \item Ein-/Ausgabe Konzept\&Komponenten
        \begin{itemize}
            \item Sensoren und Aktoren
            \item Kommunikationsschnittstellen
        \end{itemize}
    \item Speichertechnologien und Anbindung
    \item Systempartitionierung: Aufteilen der Funktionen der Komponente
    \item Logik- und Schaltungsentwurf
    \item Auswahl geeigneter Halbleitertechnologien
    \item Entwicklung von Treibersoftware
    \item Wahl eines Laufzeits-/Betriebssystems
    \item Die eigentliche Softwareentwicklung
\end{itemize}
$\Rightarrow$ Aufteilung des Entwurfs auf mehrer Entwurfsebene

\subsection{Entwurfsebenen}

