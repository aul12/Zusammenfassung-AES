\chapter{Processing Elements}
\section{Instruction Set Processor nach von Neumann (Princeton-Architektur)}
\todo{Grafik}
\subsection{Befehlszyklus}
\begin{enumerate}
    \item Befehl holen (fetch)
    \item Befehl dekodieren (decode)
    \item Operanden holen (load)
    \item Befehl ausführen (execute)
    \item Daten speichern (write back)
\end{enumerate}

\begin{table}[H]
    \centering
    \begin{tabular}{p{.5\textwidth}p{.5\textwidth}}
        \toprule
        Pro & Contra \\
        \midrule
        Analyse einfach & Auslastung gering \\
        Speicher flexibel benutzbar & von-Neumann Flaschenhals (Daten und Befehle über den selben Bus)\\
        \bottomrule
    \end{tabular}
\end{table}

\section{Harvard-Architektur}
\todo{Grafik}
\begin{table}[H]
    \centering
    \begin{tabular}{p{.5\textwidth}p{.5\textwidth}}
        \toprule
        Pro & Contra \\
        \midrule
        Auslastung & Fragmentierter Speicher \\
        Kein Flaschenhals & Analyse schwierig \\
        Schnellere Abarbeitung & Schwierig bei Datenabhängigkeiten \\
        \bottomrule
    \end{tabular}
\end{table}
